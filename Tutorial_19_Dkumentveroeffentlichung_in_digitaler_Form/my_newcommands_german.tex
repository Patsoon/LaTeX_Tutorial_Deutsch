% Einige nützliche newcommand Definitionen
% für deutsche LaTeX Texte

% Abkürzungen für 'zum Beispiel', 'unter anderem'
% etc. In diesen Abkürzungen ist das Leerzeichen
% zwischen den Buchstaben kleiner als
% normalerweise. Deswegen wird dieses
% mit '\,' anstatt mit einem space gesetzt.
% Abgeschlossen werden die Definitionen durch ein
% \xspace so dass, wenn nötig,
% im Text ein Leerzeichen nach den Makros eingefügt
% wird.
\newcommand{\zB}{z.\,B.\xspace} % zum Beispiel (z. B.)
\newcommand{\ua}{u.\,a.\xspace} % unter anderem (u. a.)
\newcommand{\uU}{u.\,U.\xspace} % unter Umständen (u. U.)

% Kurzformen für existierende, lange LaTeX Befehle:
\newcommand{\tb}{\textbackslash}

% Kommandos für Verweise (ref) Befehle:
\newcommand{\chapterref}[1]{Kapitel~\ref{#1}}
\newcommand{\sectionref}[1]{Abschnitt~\ref{#1}}
\newcommand{\subsectionref}[1]{Unterabschnitt~\ref{#1}}
\newcommand{\equationref}[1]{Gl.~(\ref{#1})}
\newcommand{\figureref}[1]{Figur~\ref{#1}}
\newcommand{\tableref}[1]{Tabelle~\ref{#1}}

% Kommandos für Mathemtikkonstrukte:

% Betragsstriche mit korrekter Größe um jedes Objekt:
\newcommand{\abs}[1]{\left|#1\right|}

% Ableitungen mit aufrecht gedruckten Differentialoperator:
\newcommand{\deriv}[2]{\frac{\mathrm{d} #1}{\mathrm{d} #2}}

% Aufrecht gedruckte Eulersche Zahl und imaginäre Einheit:
\newcommand{\euler}{\mathrm{e}}
\newcommand{\imag}{\mathrm{i}}
