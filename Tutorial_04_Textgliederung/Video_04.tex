% Das Dokument verdeutlicht die Benutzung von Textgliederungselementen

% Die Dokumentklasse bestimmt welche Textgliederungselemente zur
% Verfügung stehen. Während zum Beispiel Abschnitte in allen Klassen
% vorkommen gibt es Kapitel nur in Reports und Büchern.
\documentclass[a4paper,12pt]{scrartcl}

\usepackage[ngerman]{babel}   % Deutsche Einstellungen
\usepackage[utf8]{inputenc}   % utf-8 Eingabe
\usepackage[T1]{fontenc}      % Korrekte Behandlung von Sonderzeichen
                              % in der Ergebnis PDF-Datei.
\usepackage{csquotes}         % Für Anführungsstriche (Gänsefüßchen)

% Festlegung von Titel, Autor und Dokumentdatum
\title{Unser erstes \LaTeX{} Dokument}
\author{Thomas Erben}
\date{\today}

\begin{document}
%
% Zu Anfang des Dokuments lassen wir uns Titel Inhaltsverzeichnis
% ausgeben. Beachten Sie dass die Definition des Titels in der
% Präambel noch nicht dazu führt dass dieser auch im Dokument erscheint!
%
\maketitle
\tableofcontents
% Falls Sie wollen dass das Inhaltsverzeichnis auf eine
% eigene Seite kommt so entfernen Sie das Kommentarzeichen am
% Anfang der nächsten Zeile:
% \newpage
%

%
% Wir fangen unseren Text mit einem Absatz an:
\section{Erster Abschnitt}
Das ist unser erster Text in \LaTeX{}. Wir haben die Installation abgeschlossen
und können jetzt endlich loslegen.

% Hier beginnt ein neuer Absatz.
Dies ist der zweite Absatz.
%
% Dann noch ein Unterabschnitt
\subsection{Ein Unterabschnitt}
Neuer Text in einem Unterabschnitt.
%
% Und noch ein ganz neuer Abschnitt
\section{Der nächste Abschnitt}
Noch mehr Text.
\end{document}
