% Eine kommentierte Version der Video Datei zum
% 14. Teil des YouTube LaTeX Tutorials
% (siehe https://www.youtube.com/channel/UCgaFgieXi6HIryaFyhhzQtg)
\documentclass[12pt,a4paper]{article}

\usepackage[ngerman]{babel}
\usepackage[utf8]{inputenc}
\usepackage[T1]{fontenc}

% Das Paket 'blindtext' dient zur einfachen
% Erstellung längeren 'Dummy' Textes.
\usepackage{blindtext}

\usepackage{amsmath}
\usepackage{amssymb}

% 'newcommand' Definitionen zur Vereinfachung
% von Verweisungsbefehlen (ref-Befehle):
\newcommand{\sectionref}[1]{Abschnitt~\ref{#1}}
\newcommand{\subsectionref}[1]{Unterabschnitt~\ref{#1}}

% Beachte dass das Paket amsmath einen Befehl '\eqref'
% definiert. Dieser Befehl schreibt die Referenznummer
% in Klammern aber ohne den Text "Gl." zu setzen.
\newcommand{\equationref}[1]{Gl.~(\ref{#1})}


\begin{document}
%
\tableofcontents
%
\section{Vorwort}
Hier ist ein Vorwort.
%
\section{Dummy Text}
\label{sec:dummy}
%
% Beachten Sie das Verweisen mit direkten '\ref'-Befehlen
% (samt notwendigem Text auf das Objekt auf welches verwiesen\
% werden soll) und den einfacheren Versionen mit obigen,
% durch 'newcommand' definierte Befehle:
Zusammenhänge zwischen Masse, Energie und Beschleunigung werden anhand
Gl.~(\ref{eq:einstein}) und \equationref{eq:newton}
in \sectionref{sec:mathe} diskutiert. Hier zunächst ein wenig
\emph{dummy} Text:
 
\blindtext[3]
%.
\section{Sinnvolle Mathematik}
\label{sec:mathe}
%
Nach etwas sinnlosem Text in Abschnitt~\ref{sec:dummy} hier nun etwas
sinnvolles. In \subsectionref{subsec:nochmal_dummy} geht es dann
wieder mit \emph{Dummy}-Text weiter. Zwischen Masse $m$ und Energie
$E$ besteht nach Einstein der Zusammenhang:
%
\begin{equation}
  \label{eq:einstein}
  E=mc^{2}
\end{equation}
Laut Newton ist die Kraft $\vec{F}$ proportional zur Beschleunigung
$\vec{a}$:
%
\begin{equation}
  \label{eq:newton}
  \vec{F}=m\vec{a}.
\end{equation}
%
\subsection{Wieder nur Dummy ....}
\label{subsec:nochmal_dummy}
%
\blindtext
%
\end{document}
