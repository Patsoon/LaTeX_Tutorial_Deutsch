\chapter{Der kleine Hobbit}
%
\label{chap:hobbit}
Nachdem wir in \chapterref{chap:latex_elemente} einige \LaTeX{}-Elemente
früherer Videos zusammengefasst haben schreiben wir hier etwas über den
kleinen Hobbit. In \chapterref{chap:mathematik} geht es anschließend
mit Mathematik weiter.

Hier noch etwas mehr Text. Hier tippe ich jetzt weiter.
%
\section{Eine unvorhergesehene Gesellschaft}

\subsection*{Die Hobbithöhle}
In einer Höhle in der Erde, da lebte ein \textbf{Hobbit}. Nicht in
einem schmutzigen, nassen Loch, in das die Enden von irgendwelchen
Würmern herabbaumelten und das nach Schlamm und Moder roch. Auch nicht
etwa in einer trockenen Kieshöhle, die so kahl war, dass man sich
nicht einmal niedersetzen oder gemütlich frühstücken konnte. Es war
eine \underline{Hobbithöhle}, und das bedeutet Behaglichkeit.

Diese Höhle hatte eine kreisrunde Tür wie ein Bullauge. Sie war grün
gestrichen, und in der Mitte saß ein glänzend gelber Messingknopf. Die
Tür führte zu einer röhrenförmig langen Halle, zu einer Art Tunnel,
einem Tunnel mit getäfelten Wänden.

\subsection*{Ein guter Morgen (?)}
Alles, was also der keineswegs misstrauische Bilbo an diesem Morgen
sah, war ein kleiner, alter Mann mit einem Stab, hohem, spitzem blauen
Hut, einem langen, grauen Mantel, mit einer silbernen Schärpe, über
die sein langer, silberner Bart hing, ein kleiner, alter Mann mit
riesigen schwarzen Schuhen.

\enquote{Guten Morgen}, sagte Bilbo, und er meinte es ehrlich. Die
Sonne schien, und das Gras war grün. Aber Gandalf schaute ihn scharf
unter seinen buschigen Augenbrauen hervor an.

\enquote{Was meint Ihr damit?} fragte er. 
\begin{itemize}
  \item \enquote{\emph{Wünscht Ihr mir einen guten Morgen?}}

  \item \enquote{\emph{Oder meint Ihr, dass dies ein guter Morgen
        ist, gleichviel, ob ich es wünsche oder nicht?}}
  \item \enquote{\emph{Meint Ihr, dass Euch der Morgen gut bekommt?}}
  \item \enquote{\emph{Oder dass dies ein Morgen ist, an dem man gut
        sein muss?}}
\end{itemize}

\enquote{Alles auf einmal}, sagte Bilbo. \enquote{Wie heißt Ihr
  eigentlich?} fragte der Hobbit. \enquote{Ich bin Gandalf, und Gandalf,
denkt nur, das bin ich!} antwortete der Zauberer.
%
\section{Die Zwerge}
\label{sec:zwerge}
Mit dem Namen Gandalf fiel bei Bilbo der Groschen. Der Zauberer war
früher oft zu Gast bei den Hobbits. Er hatte seinem Grossvater vor
Ewigkeiten ein paar magische Diamantenklammern geschenkt und die
Hobbits zur Sommersonnenwende stets mit beeindruckenden Feuerwerken
erfreut. Nach einer Weile lud Bilbo den Zauberer für den nächsten Tag
zum Tee ein und dieser verschwand so schnell wie er gekommen war.
 
Bevor Gandalf am folgenden Nachmittag erschien, wurde der arme Hobbit
von 13 ungebetenen Gästen, es waren Zwerge, heimgesucht. Ihre Namen
finden sich, zusammen mit einigen Zusatzinformationen, in
Tabelle~\ref{tab:zwerge}.
% 
\begin{table}
  \centering
  % Die folgenden zwei Befehle erzeugen eine Tabellenüberschrift
  \captionabove{Die dreizehn Zwerge}
  \label{tab:zwerge}
  \begin{tabular}{l||l|l|l|l|l}
  Name   & Bart & Gürtel & Kapuze & Instrument & Sonstiges \\
  \hline
  Dwalin & blau & gold & dunkelgrün & Bratsche & \\
  Balin & weiß & & purpurrot & Bratsche & \\
  Kili & gelb & silber & blau & Fiedel & Werkzeug \\
  Fili & gelb & silber & blau & Fiedel & Spaten \\
  Dori & & gold & purpurrot & Flöte & \\
  Nori & & gold & purpurrot & Flöte & \\
  Ori & & gold & grau & Flöte & \\
  Oin & & silber & braun & & \\
  Gloin & & silber & silber & & \\
  Bifur & & & gelb & Klarinette & \\
  Bofur & & & gelb & Klarinette & \\
  Bombur & & & blaßgrün & Trommel & fett \\
  Thorin & & & himmelblau mit & Harfe & sehr berühmt \\
         & & & silberner Schärpe & &
  \end{tabular}
\end{table}
%
