% Eine dokumentierte Version der Video Datei zum siebzehnten
% Teil des LaTeX Tutorials.

% Die Option 'bibtotoc' in den Optionen der Dokumentklasse
% sorgt dafür dass das Literaturverzeichnis im Inhaltsverzeichnis
% erscheint.
\documentclass[12pt,a4paper,bibtotoc]{scrartcl}

\usepackage[ngerman]{babel}
\usepackage[utf8]{inputenc}
\usepackage[T1]{fontenc}
\usepackage{csquotes}

% Das Paket biblatex ist für Literaturverweise zuständig.
\usepackage[backend=biber,
              % Wir benuten biber für Literaturverzeichnisse
              % Ohne diese Optionen werden Literaturdatenbanken
              % im utf-8 Format nicht korrekt behandelt.
            style=alphabetic % Zitierstil
           ]{biblatex}

% Das Literaturdatenbankfile (hier 'literatur.bib') muss
% sich im selben Verzeichnis wie das LaTeX Dokument befinden.
\addbibresource{literatur.bib}

\begin{document}
%
\tableofcontents
\section{Literaturzitate}
% Literaturverweise werden durch den Befehl 'cite' und die
% Zeichenkette, die das zu zitierende Dokument markiert, gesetzt.
% Die Zeichenkette ist identisch zu der im Literaturdatenbankfile.
Studienanfänger der Physik benutzen oft \cite{gerthsen2013physik} als
Einführungstext in die Experimentalphysik. Für \LaTeX{} verwenden wir
den Onlinetext \cite{daniel2015l2lurz}. Während oder nach der
Masterarbeit schreiben wir vielleicht auch mal eine Veröffentlichung
wie hier in \cite{heymans2006shear}.
%
% Setzen Sie den 'printbibliography' Befehl an die Stelle wo
% Sie das Literaturverzeichnis im Ausgabedokument sehen wollen.
\printbibliography
\end{document}
